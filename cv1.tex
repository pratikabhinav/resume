%
% LaTeX source of my resume
% =========================
%
% Heavily commented to to fit even LaTeX beginners (hopefully).
%
% See the `README.md` file for more info.
%
% This file is licensed under the CC-NC-ND Creative Commons license.
%


% Start a document with the here given default font size and paper size.
\documentclass[10pt,a4paper]{article}

% Set the page margins.
\usepackage[letterpaper,margin=0.55in]{geometry}

\usepackage{hyperref}

% Setup the language.
\usepackage[english]{babel}
\hyphenation{Some-long-word}

% Makes resume-specific commands available.
\usepackage{resume}

\usepackage{enumitem}

\begin{document}  % begin the content of the document
\sloppy  % this to relax whitespacing in favour of straight margins


% title on top of the document
\maintitle{Abhinav Pratik}{}{}

\nobreakvspace{0.3em}  % add some page break averse vertical spacing

% \noindent prevents paragraph's first lines from indenting
% \mbox is used to obfuscate the email address
% \sbull is a spaced bullet
% \href well..
% \\ breaks the line into a new paragraph
\noindent\href{mailto:pratikabhinav@gmail.com}{pratikabhinav\mbox{}@\mbox{}ufl.edu}\sbull (+1) 352-871-7127 \sbull \href{https://github.com/pratikabhinav}{github.com/pratikabhinav}\\
3700 SW 27th Street\sbull
Apartment C201\sbull Gainesville\sbull Florida 32608
\spacedhrule{0.5em}{-0.4em}

\roottitle{Objective}
\bodytext{I am looking for a software/web developer internship that gives me exposure to solving real world business problems
through large-scale software design.}

\spacedhrule{0.5em}{-0.4em}
\roottitle{Experience}

\headedsection  % sets the header for the section and includes any subsections
  {\href{}{\textbf{Zenterior Technology Solutions Pvt. Ltd.}}}
  {\textsc{Chennai, India}}
  {%
  \headedsubsection
    {Intern - Software Developer}
    {May~'17 -- June~'17}
    {\bodytext{
    \begin{itemize}
    \item Ported the company's website to both Android and iOS platforms as an app single-handedly. I used hybrid mobile app development using ionic framework and Cordova plugins to complete the project. Ionic and Cordova allows us to leverage web technologies to create native-looking mobile apps. 
    \end{itemize}
}}
}
\headedsection  % sets the header for the section and includes any subsections
  {\href{}{\textbf{cBioPortal}}}
  {\textsc{Google Summer of Code Project}}
  {%
  \headedsubsection
    {Full Stack Developer}
    {March~'17 -- April~'17}
    {\bodytext{
    \begin{itemize}
    \item For my GSoC 2017 project proposal, I implemented an Interactive web tour across cBioPortal, that guides users through the different features of the website and explains with an easy step-by-step procedure, how to use them. Although this was a significant project and was appreciated by the mentors, due to lack of resources they could not go forward with it. You can find the entire project proposal at \href{https://goo.gl/oAv5EP}{https://goo.gl/oAv5EP}\\
    \end{itemize}
}}
}
\spacedhrule{0.5em}{-0.4em}
\roottitle{Education}

\headedsection
  {\href{}{Herbert Wertheim College of Engineering, University of Florida}}
  {\textsc{Gainesville, Florida}} {%
  \headedsubsection
    {Master of Science in Computer Science }
    {August~'18 -- Ongoing} {}
}

\headedsection
  {\href{}{Herbert Wertheim College of Engineering, University of Florida}}
  {\textsc{Gainesville, Florida}} {%
  \headedsubsection
    {Computer Science Senior Certificate Student}
    {January~'18 -- May~'18} {}
}

\headedsection
  {\href{}{Amity School of Engineering and Technology (ASET), Amity University}}
  {\textsc{Noida, Uttar Pradesh}} {%
  \headedsubsection
    {Bachelor of Technology (B.Tech.)~in Computer Science}
    {July~'14 -- May~'18} {}
}



\headedsection
{\href{}{Delhi Public School}}
{\textsc{Patna, India}} {%
  \headedsubsection
  {Senior Secondary (AISSCE, CBSE)}
  {April~'14} {}

  \headedsubsection
  {Secondary (AISSE, CBSE)}
  {April~'12} {}
  
}

\spacedhrule{0.5em}{-0.4em}

\roottitle{Selected Projects}
\vspace{0.15cm}
\begin{itemize}[labelindent=1.5em,labelsep=-0.3cm,leftmargin=*]

\item \headedsubsection
{Chord Protocol Implementation}{{{November~'18}}}
{\bodytext{Chord is a protocol and algorithm for a peer-to-peer distributed hash table. The goal of the project was to implement the network join and routing as described in the Chord paper (Section 4) and encode the simple application that associates a key (same as the ids used in Chord) with a string.}}

\item \headedsubsection
{A Gossip Simulator}{{{October~'18}}}
{\bodytext{Goal of the project was to implement gossip and push sum algorithms and observe its behavior for various network topologies. Topologies implemented were : Full (full), 3D (3d), Random 2D (rand2d), Imperferct 2D (imp2d), Line (line), Imperfect line (impline).}}

\item \headedsubsection
{A Real-Time Social Blog}{{{Undergrad project}}}
{\bodytext{A micro blog with trending posts and upvotes feature for the students of my college that would let them stay in touch with all the events in the entire University. The project was developed using MEAN stack. }}

\item \headedsubsection
{A Movies and TV Shows Information Website}{{{Independent project}}}
{\bodytext{A JavaScript/jQuery app designed using Bootstrap CSS and HTMl, that fetches movie / tv show data from the OMDb API. The requests to the API are made using Axios NPM, which is a promise based HTTP client for the browser and node.js.}}

\item \headedsubsection
{A Chat Web App}{{{Independent project}}}
{\bodytext{Created a material designed, minimal chat web app, using the Firebase platform that syncs data using the Firebase Realtime Database and Cloud Storage. Users need to sign in using Firebase Google Auth and can interact with each other through text messages and images.}}

\item \headedsubsection
{A Pomodoro Timer}{{{Independent project}}}
{\bodytext{Built a versatile Pomodoro timer \textbf{Chrome extension} to improve productivity. The extension also has options for a custom timer and a "50-10" timer. Also created a web version of the Pomodoro timer using HTML, CSS and jQuery. }}

\item \headedsubsection  % sets the header for the section and includes any subsections
  {{\normalfont Solved around 150 problems on various online programming platforms like Timus Online Judge (acm.timu.ru), Codechef, Spoj, InterviewBit, and from the book Elements of Programming Interviews. }}{{{Independent Project}}}
  {\bodytext{Language used :~C++}}
  %{\textsc{}}
 

\end{itemize}

\spacedhrule{0.5em}{-0.4em}
\roottitle{Skills}
{\bodytext{\textbf{Programming Languages}:~C/C++, Java, Elixir}}
{\bodytext{\textbf{Others}:~SQL, Python, Git, \LaTeX. HTML, CSS, Javascript, Angular, jQuery, Bootstrap, Ionic, Cordova, MEAN stack, Wordpress  }}
{\bodytext{\textbf{Platforms}:~Linux and Windows  }}


\spacedhrule{0.5em}{-0.4em}

\roottitle{Trainings}
\vspace{0.15cm}
\begin{itemize}[labelindent=1.5em,labelsep=-0.3cm,leftmargin=*]

\item \headedsubsection 
    {Machine Learning}{{{January~'19 -- Ongoing}}}
  %{\textsc{}}
    {\bodytext{Prof.~Andrew Ng, Stanford University}}

\item \headedsubsection 
    {Algorithms : Design and Analysis}{{{Jan~'17 -- March~'17}}}
  %{\textsc{}}
    {\bodytext{Prof.~Tim Roughgarden, Stanford University}}
    
\item \headedsubsection 
    {Full Stack Web Developer Specialization}{{{May~'16 -- January~'17}}}
      %{\textsc{}}
    {\bodytext{Prof.~Jogesh K. Muppala, Prof.~David Rossiter, The Hong Kong University of Science and Technology}}
    
\item \headedsubsection 
    {Server-side Development with NodeJS}{{{December~'16 -- January~'17}}}
      %{\textsc{}}
    {\bodytext{Prof.~Jogesh K. Muppala, The Hong Kong University of Science and Technology}}

\item \headedsubsection 
    {Multiplatform Mobile App Development with Web Technologies}{{{November~'16 -- December~'16}}}
      %{\textsc{}}
    {\bodytext{Prof.~Jogesh K. Muppala, The Hong Kong University of Science and Technology}}
    
\item \headedsubsection 
    {Front-End JavaScript Frameworks: AngularJS}{{{September~'16 -- October~'16}}}
      %{\textsc{}}
    {\bodytext{Prof.~Jogesh K. Muppala, The Hong Kong University of Science and Technology}}
    
\item \headedsubsection 
    {Front-End Web UI Frameworks and Tools}{{{July~'16 -- August~'16}}}
      %{\textsc{}}
    {\bodytext{Prof.~Jogesh K. Muppala, The Hong Kong University of Science and Technology}}
    

\item \headedsubsection 
    {HTML, CSS and Javascript}{{{May~'16 -- June~'16}}}
  %{\textsc{}}
    {\bodytext{Prof.~David Rossiter, The Hong Kong University of Science and Technology}}
    

\item \headedsubsection  % sets the header for the section and includes any subsections
  {{\normalfont Advanced's C++ through Bo Qian's YouTube channel. Bo Qian is a Software Engineer at Apple with with over 9 years experience in Object oriented software development. He masters firm command over C++ }}{{{November~'16 -- Jan~'17}}}
  %{\textsc{}}
  
\end{itemize}


\spacedhrule{0.5em}{-0.4em}

\roottitle{Additional Activities}
\vspace{0.15cm}
\begin{itemize}[labelindent=1.5em,labelsep=-0.3cm,leftmargin=*]

\item \headedsubsection 
{\href{}{\normalfont \textbf{One of 86 students} to be selected for the 6th Undergraduate Summer School conducted by the Department of Computer Science and Automation at
\textbf{Indian Institute of Science, Bangalore}. Also finished \textbf{2nd} in the \textbf{on-site hackathon}. 
}}{{{}}}


\item \headedsubsection 
{\href{}{\normalfont MOOC (Massive Open Online Courses) enthusiast with \textbf{8 Certificates} of Accomplishments on \textbf{Coursera}.
}}{{{}}}

\item \headedsubsection 
{\href{}{\normalfont Served as a \textbf{Student Mentor} for the CSE Department at Amity University. Helped new students get acquainted with the college environment and studies. Also held several positions of responsibility during various events.
}}{{{ 2015-17}}}

\item \headedsubsection {\normalfont \textbf{Captain} of the \textbf{High School Soccer Team} at various Inter-school tournaments.}{{{ 2012-14}}}

\end{itemize}

\end{document}